%!TEX root=../document.tex

\section{Ergebnisse}
\label{sec:Ergebnisse}

GitHub Repository Link: https://github.com/Pwngu/Java-WebApp

\subsection{Projekt Setup \& Deployment}
Da ich das Projekt mit Gradle umsetzen wollte, habe ich ein Github Repository mit Tutorial gefunden, welches das aufsetzen eines einfaches Projektes gut erklärt. \cite{gradle-jersey-tutorial}.\\
Für das Tutorial wurde der Tomcat Application Server \cite{tomcat-hp} verwendet. 

\subsubsection{Probleme beim Deployment}

Beim deployment auf den Tomcat Server ist ein ziemliches Problem aufgetreten; der Tomcat Server scheint den konfigurierten Pfad nicht zu registrieren. Bei folgender Konfiguration (mit Projektnamen \texttt{java-web}):

\begin{lstlisting}[style=XML, caption=web.xml]
<?xml version="1.0" encoding="UTF-8"?>
<web-app xmlns:xsi="http://www.w3.org/2001/XMLSchema-instance"
         xmlns="http://java.sun.com/xml/ns/javaee"
         xmlns:web="http://java.sun.com/xml/ns/javaee/web-app_2_5.xsd"
         xsi:schemaLocation="http://java.sun.com/xml/ns/javaee http://java.sun.com/xml/ns/javaee/web-app_2_5.xsd"
         id="WebApp_ID" version="2.5">
    <display-name>Java Web</display-name>
    <servlet>
        <servlet-name>jersey</servlet-name>
        <servlet-class>com.sun.jersey.spi.container.servlet.ServletContainer</servlet-class>
        <init-param>
            <param-name>com.sun.jersey.config.property.packages</param-name>
            <param-value>at.tgm.ablk</param-value>
        </init-param>

        <init-param>
            <param-name>com.sun.jersey.api.json.POJOMappingFeature</param-name>
            <param-value>true</param-value>
        </init-param>
        <load-on-startup>1</load-on-startup>
    </servlet>
    <servlet-mapping>
        <servlet-name>jersey</servlet-name>
        <url-pattern>/rest/*</url-pattern>
    </servlet-mapping>
</web-app>
\end{lstlisting}

\begin{lstlisting}[style=Java, caption=Java Testfile]
package at.tgm.ablk;

import javax.ws.rs.GET;
import javax.ws.rs.Path;
import javax.ws.rs.Produces;
import javax.ws.rs.core.MediaType;

@Path("login")
public class LoginResource {

    @GET
    @Path("/do")
    @Produces(MediaType.TEXT_PLAIN)
    public String getLogin() {
        return "hello";
    }
}
\end{lstlisting}

sollte das Interface unter \texttt{localhost:8080/java-web/rest/login/do} erreichbar sein, jedoch bekomme ich immer, auch bei mehreren Variationen des Pfades eine 404 Response.\\
Auch in den Tomcat log files ließ sich nichts auffälliges finden.\\
Aufgrund dieses Fehlers konnte ich das Programm nicht wirklich testen.

\subsection{Datenbank}

Für die Aufgabe habe ich mich für eine \textit{SQLite} Datenbank entschieden und verwende einen dazugehörigen JDBC Treiber \cite{sqlite-jdbc}. Dieser erstellt automatisch ein Datenbankfile an der gegebenen Stelle.

\begin{lstlisting}[style=Java, caption=Aufbau der SQLite Datenbenkverbindung]
try {

	Class.forName("org.sqlite.JDBC");
    connection = DriverManager.getConnection("jdbc:sqlite:user.db");
    connection.setAutoCommit(false);
} catch(Exception ex) {

	LOG.fatal("Error while connecting to Database", ex);
   	System.exit(
}

LOG.info("Opened database connection successfully");
\end{lstlisting}

\subsection{REST Interface}

Sämtliche Kommunikation über das REST-Interface verläuft über JSON. Um JSON Requests akzeptieren zu können, ist es notwendig zuerst eine Klasse, welches das zu empfangende JSON repräsentiert, zu erstellen (Hier \texttt{at.tgm.ablk.rest.User}).

\subsubsection{Registrierung}

Bei der Registrierung muss eine JSON - POST request an den Server geschickt werden, jeweils mit der E-Mail und dem Passwort. Wenn die Registrierung erfolgreich war, wird mit einer 201 response geantwortet, ansonsten mit 401.\\
Derzeit wird noch nicht überprüft, ob die übergebene E-Mail bereits registriert ist.

\begin{lstlisting}[style=Java, caption=REST User Registration]
@POST
@Consumes({"application/json"})
@Path("/register")
public Response register(final User user) {

    if(user.email == null || user.password == null)
        return Response.status(401).header("message", "Invalid registration data").build();

    try(Statement stmt = database.getConnection().createStatement();
        ResultSet res = stmt.executeQuery(String.format("INSERT INTO web_user (email, password)" +
                                                        "VALUES (%s, %s)", user.email, user.password))) {

        database.getConnection().commit();
        return Response.status(201).header("message", "Successfully registered user").build();

    } catch(SQLException ex) {
        LOG.error("Database error while registering", ex);
        return Response.status(500).header("message", "SQL Error, check server log").build();

    } catch(Exception ex) {
        LOG.error("Error while registering", ex);
        return Response.status(500).header("message", "Exception, check server log").build();
    }
}
\end{lstlisting}

\subsubsection{Login}

Beim Login wird, genauso wie bei der Registrierung, eine JSON - POST request erwartet. Hier muss das JSON file auch eine E-Mail Adresse und ein Passwort enthalten, danach wird überprüft, ob der angegebene Nutzer existiert, wenn ja, wird das Passwort überprüft und bei Erfolg ein 200 response mit Willkommensnachricht verschickt.

\begin{lstlisting}[style=Java, caption=REST User Login]
@POST
@Consumes({"application/json"})
@Path("/login")
public Response login(final User user) {

    if(user.email == null || user.password == null)
        return Response.status(403).header("message", "Invalid credentials").build();

    try(Statement stmt = database.getConnection().createStatement();
        ResultSet res = stmt.executeQuery(String.format("SELECT email, password FROM web_user " +
                                                        "WHERE email = %s", user.email))) {

        while(res.next()) {

            if(res.getString("password").equals(user.password))
                return Response.status(200).header("message", "Successfully logged in").build();
        }

        // not registered
        return Response.status(403).header("message", "Invalid credentials").build();

    } catch(SQLException ex) {
        LOG.error("Database error while logging in", ex);
        return Response.status(500).header("message", "SQL Error, check server log").build();

    } catch(Exception ex) {
        LOG.error("Error while logging in", ex);
        return Response.status(500).header("message", "Exception, check server log").build();
    }
}
\end{lstlisting}



