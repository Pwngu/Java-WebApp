%!TEX root=../document.tex

\section{Einführung}
Diese Übung zeigt die Anwendung von mobilen Diensten in Java.

\subsection{Ziele}

Das Ziel dieser Übung ist eine Webanbindung zur Benutzeranmeldung in Java umzusetzen. Dabei soll sich ein Benutzer registrieren und am System anmelden können.

Die Kommunikation zwischen Client und Service soll mit Hilfe von JAX-RS (Gruppe1) umgesetzt werden.

\subsection{Voraussetzungen}
	\begin{itemize}
		\item Grundlagen Java und Java EE
		\item Verständnis über relationale Datenbanken und dessen Anbindung mittels JDBC oder ORM-Frameworks
		\item Verständnis von Restful Webservices
	\end{itemize}

\subsection{Aufgabenstellung}

Es ist ein Webservice mit Java zu implementieren, welches eine einfache Benutzerverwaltung implementiert. Dabei soll die Webapplikation mit den Endpunkten /register und /login erreichbar sein.

\textbf{Registrierung}
Diese soll mit einem Namen, einer eMail-Adresse als BenutzerID und einem Passwort erfolgen. Dabei soll noch auf keine besonderen Sicherheitsmerkmale Wert gelegt werden. Bei einer erfolgreichen Registrierung (alle Elemente entsprechend eingegeben) wird der Benutzer in eine Datebanktabelle abgelegt.

\textbf{Login}
Der Benutzer soll sich mit seiner ID und seinem Passwort entsprechend authentifizieren können. Bei einem erfolgreichen Login soll eine einfache Willkommensnachricht angezeigt werden.

Die erfolgreiche Implementierung soll mit entsprechenden Testfällen dokumentiert werden. Es muss noch keine grafische Oberfläche implementiert werden! Verwenden Sie auf jeden Fall ein gängiges Build-Management-Tool..
\clearpage

\subsection{Quellen}

"Android Restful Webservice Tutorial – Introduction to RESTful webservice – Part 1"; Posted By Android Guru on May 1, 2014; online: http://programmerguru.com/android-tutorial/android-restful-webservice-tutorial-part-1/
"REST with Java (JAX-RS) using Jersey - Tutorial"; Lars Vogel; Version 2.5; 15.12.2015; online: http://www.vogella.com/tutorials/REST/article.html
"O Java EE 7 Application Servers, Where Art Thou? Learn all about the state of Java EE app servers, a rundown of various Java EE servers, and benchmarking."; by Antonio Goncalves; Java Zone; Feb. 10, 2016; online: https://dzone.com/articles/o-java-ee-7-application-servers-where-art-thou
"Heroku makes it easy to deploy and scale Java apps in the cloud"; online: https://www.heroku.com/

Bewertung: 16 Punkte
- Aufsetzen einer Webservice-Schnittstelle (4 Punkte)
- Registrierung von Benutzern mit entsprechender Persistierung (4 Punkte)
- Login und Rückgabe einer Willkommensnachricht (3 Punkte)
- AcceptanceTests (3 Punkte)
- Protokoll (2 Punkte)
\clearpage
